\documentclass[8pt,portrait]{article}
%\usepackage{pslatex}
\usepackage{graphicx}
\usepackage{multicol}
\usepackage{calc}
\usepackage{color}

\usepackage{hyperref}
\usepackage[]{url}
\hypersetup{%
  colorlinks=true,  
  linkcolor = black,
  citecolor = black, 
  urlcolor = blue, 
  pdfpagemode=UseNone, 
  pdfstartview=FitH,
  pdfauthor= {Armando Fandango},
  pdftitle= Reference Card
}

\urlstyle{same}


% Turn off header and footer
\pagestyle{empty}

%\setlength{\leftmargin}{0.75in}
\setlength{\oddsidemargin}{-0.75in}
\setlength{\evensidemargin}{-0.75in}
\setlength{\textwidth}{7.9in}


\setlength{\topmargin}{-0.2in}
\setlength{\textheight}{9.8in}
\setlength{\headheight}{0in}
\setlength{\headsep}{0in}

\pdfpageheight\paperheight
\pdfpagewidth\paperwidth

% Redefine section commands to use less space
\makeatletter
\renewcommand\section{\@startsection{section}{1}{0mm}%
                                     {-24pt}% \@plus -12pt \@minus -6pt}%
                                     {0.5ex}%
                                {\color{blue}\normalfont\large\bfseries}}
\makeatother

% Don't print section numbers
\setcounter{secnumdepth}{0}

\setlength{\parindent}{0pt}
\setlength{\parskip}{0pt}

\newcommand{\code}{\texttt}
\newcommand{\bcode}[1]{\texttt{\textbf{#1}}}
\newcommand\F{\code{FALSE}}
\newcommand\T{\code{TRUE}}

%\newcommand{\hangpara}[2]{\hangindent#1\hangafter#2\noindent}
%\newenvironment{hangparas}[2]{\setlength{\parindent}{\z@}\everypar={\hangpara{#1}{#2}}}

\newcommand{\describe}[1]{\begin{description}{#1}\end{description}}


% -----------------------------------------------------------------------

\begin{document}

%\raggedright
\footnotesize
\begin{multicols*}{2}

% multicol parameters
% These lengths are set only within the two main columns
%\setlength{\columnseprule}{0.25pt}
\setlength{\premulticols}{1pt}
\setlength{\postmulticols}{1pt}
\setlength{\multicolsep}{1pt}
\setlength{\columnsep}{2pt}

\begin{center}
     {\Large{\textbf{\color{blue}UCF ARCC Stokes Reference Card}}} \\
 \today
\end{center}

Created and maintained by \href{mailto:armando@ucf.edu}{Armando Fandango}, available from \href{https://github.com/HPCGroup/UCF-ARCC-refcard}{GitHub}. Feel free to contact him for contributions.

\section{Getting help}
\everypar={\hangindent=9mm}
ARCC helpdesk: \href{mailto:request-stokes@ist.ucf.edu}{request-stokes@ist.ucf.edu}\\

\section{Stokes - Logging In}
\everypar={\hangindent=9mm}

\bcode{ssh -Y -i path\_to\_your\_ssh\_key your\_username\@stokes.ist.ucf.edu}\\ For Mac/Unix/Windows based system with ssh client

\bcode{nano \textasciitilde{}/.ssh/config} For setting up personal SSH configuration\\
\bcode{Host stokes\\
     Hostname stokes.ist.ucf.edu\\
     user your\_username\\
     IdentityFile absolute\_path\_to\_ssh\_key}\\


\section{Stokes - Job Queues}
\everypar={\hangindent=9mm}

\begin{tabular}{@{}l@{\ }l@{\ }l@{\ }l}
\code{queue} & Max Cores & Max Nodes & Max Time\\
\code{batch64} & 512 & 64 & 240:00:00 \\
\code{batch98} & 784 & 98 & 48:00:00 \\
\code{batch224} & 2688 & 224 & 36:00:00 \\

\end{tabular}\\




%\section{Job Submission - Torque}
%\begin{tabular}{@{}l@{\ }l}
%\code{blah} & blah blah blah\\
%\code{blah} & blah blah blah\\
%\code{blah} & blah blah blah\\
%\code{blah} & blah blah blah\\
%\code{blah} & blah blah blah\\
%\code{blah} & blah blah blah\\
%\end{tabular}

\section{SLURM - Job Submission}
\everypar={\hangindent=9mm}

\code{sbatch} Submit script for later execution
\describe{
\itemsep=0pt\parskip=0pt
    \item{\code{--array=<indexes>}} {Job Array Specification}
}

\code{srun} Create a job allocation and launch a job step
\describe{
\itemsep=0pt\parskip=0pt
 \item{\code{--label}} {Prepend task ID to output}
}

\code{salloc} Create job allocation and start a shell\\

\textbf{Options common across sbatch, srun and salloc:}
\describe{
\itemsep=0pt\parskip=0pt
    \item{\code{--account=<name>}} {Account to be charged for resources used}
    \item{\code{--begin=<time>}} {Initiate job after specified time}
    \item{\code{--constraint=<features>}} {Required node features}
    \item{\code{--cpu\_per\_task=<count>}} {Number of CPU's required per task}
    \item{\code{--dependency=<state:jobid>}} {defer until jobid reaching specified state}
    \item{\code{--error=<filename>}} {File to store job errors}
    \item{\code{--exclude=<hostnames>}} {Hosts to be excluded from job allocation}
    \item{\code{--export=<name[=value]>}} {Export specified environment variable name}
    \item{\code{--gres=<name[:count]>}} {Generic resources required for the job}
    \item{\code{--input=<name>}} {File for input job data}
    \item{\code{--job-name=<name>}} {A name for the job}
} 

\section{SLURM - Job Management}

\code{sattach}  Connect stdin/out/err for an existing job or step

\code{scancel}  Cancel jobs or steps 
\describe{
\itemsep=0pt\parskip=0pt
    \item{\code{--account=<name>}} {Operate only on jobs for the specified account}
    \item{\code{--name=<name>}} {Operate only on the specified job(s)} 
    \item{\code{--partition=<name(s)>}} {Operate only on the jobs in specified partition(s)}
    \item{\code{--qos=<name>}} {Operate only on the jobs using specified QoS}
    \item{\code{--reservation=<name>}} {Operate only on specified reservation}
    \item{\code{--state=<name(s)>}} {Operate only on job in specified state}
    \item{\code{--user=<name>}} {Operate only on jobs for the specified user}
    \item{\code{--nodelist=<name(s)>}} {Operate only on jobs using the specified node(s)}
}
    
\code{sbcast}  Transfer file to compute node(s) allocated to a job 

\code{srun\_cr}  Wrapper for srun to support Berkeley checkpoint/restart 

\code{strigger} Event trigger management  create, destroy or list event triggers\\

\columnbreak

\section{SLURM - Get Information}
\everypar={\hangindent=9mm}

\code{sinfo}   System status - nodes, queues

\code{squeue}  Job and Step status
\describe{
\itemsep=0pt\parskip=0pt
    \item{\code{--account=<name>}} {Operate only on jobs for the specified account}
    \item{\code{--jobs=<job id list>}} {Operate only on the specified job(s)} 
    \item{\code{--name=<job name(s)>}} {Operate only on the specified job(s)} 
    \item{\code{--partition=<name(s)>}} {Operate only on the jobs in specified partition}
     \item{\code{--priority}} {Sort jobs by priority}
    \item{\code{--qos=<name>}} {Operate only on the jobs using specified QoS}
    \item{\code{--reservation=<name>}} {Operate only on specified reservation}
    \item{\code{--state=<name(s)>}} {Operate only on job in specified state(s)}
    \item{\code{--users=<name(s)>}} {Operate only on jobs for the specified user(s)}
    \item{\code{--nodelist=<name(s)>}} {Operate only on jobs using the specified node(s)}
}
\code{smap}  Status with curses-based GUI

\code{sview}  Status with GTK-based GUI

\code{sacct}  Accounting info by job and step 

\code{sstat}  Accounting info about currently running jobs and steps 

\code{sreport}  Resource usage by cluster, partition, user, account etc. 

\section{SLURM - Environment Variables}
\begin{tabular}{@{}l@{\ }l}
\code{SLURM\_ARRAY\_JOB\_ID} & Job Id in Array\\
\code{SLURM\_ARRAY\_TASK\_ID} & Task Id in Array\\
\code{SLURM\_CPUS\_PER\_TASK} & No. of CPUs per task\\
\code{SLURM\_JOB\_ACCOUNT} & Account Name\\
\code{SLURM\_JOB\_ID} & Job Id\\
\code{SLURM\_JOB\_NAME} & Job Name\\
\code{SLURM\_JOB\_NODELIST} & Names of nodes allocated\\
\code{SLURM\_JOB\_NUM\_NODES} & No. of nodes allocated\\
\code{SLURM\_JOB\_PARTITION} & Partition running the job\\
\code{SLURM\_JOB\_UID} & User Id of job owner\\
\code{SLURM\_JOB\_USER} & User name of job owner\\
\code{SLURM\_NPROCS} & Number of cores allocated\\
\code{SLURM\_PROCID} & Task Id (MPI Rank)\\
\code{SLURM\_STEP\_ID} & Step Id in a Job\\
\code{SLURM\_STEP\_NUM\_TASKS} & No. of tasks (MPI Ranks)\\
\end{tabular}

\section{SLURM - Administration}
\everypar={\hangindent=9mm}
\code{scontrol} Admin tool to view/update system, job, step, partition, reservation status

\code{sacctmgr}  DB Admin tool to add/delete clusters, accounts, users and get/set resource limits, fair-share allocations etc. 

\code{sprio}  View factors comprising job's priority

\code{sshare}  View current hierarchical fair-share information 

\code{sdiag}  View statistics about scheduling module

\section{Torque/MOAB - User Commands}

\begin{tabular}{@{}l@{\ }l}
\code{myquota} & Check your disk usage\\
\code{myusage} & Check your hours usage\\
\code{qsub} & submit job\\
\code{qstat | showq} & status of queue\\
\code{pbsnodes | checknode} & status of node\\
\code{checkjob} & status of node\\
\code{mdiag} & \\
\end{tabular}

\section{StarCCM}
Make sure the following lines are added to your ~/.ssh/config file:\\
\bcode{Host *\\
               StrictHostKeyChecking no\\
               UserKnownHostsFile=/dev/null}

\section{Matlab}

Compile: Get an interactive node, load Matlab module and then compile the Matlab code.\\
Run: Submit the compiled code as a SLURM job.

\everypar={\hangindent=9mm}

\code{mcc -m hello.m} Compile matlab code as standalone C application

\code{mcc -p hello.m} Compile matlab code as standalone C++ application

\end{multicols*}
\end{document}

%    \item{\code{}} {}
%    \item{\code{}} {} 
%    \item{\code{}} {}
%    \item{\code{}} {}
%    \item{\code{}} {}
%    \item{\code{}} {}
%    \item{\code{}} {}
%    \item{\code{}} {}
%    \item{\code{}} {}
%    \item{\code{}} {}
%    \item{\code{}} {}
%    \item{\code{}} {}
%    \item{\code{}} {}
%    \item{\code{}} {}  
%    \item{\code{}} {}
%    \item{\code{}} {}
%    \item{\code{}} {}
%    \item{\code{}} {}
%    \item{\code{}} {}
%    \item{\code{}} {}
%    \item{\code{}} {}
%    \item{\code{}} {}
%    \item{\code{}} {}
%    \item{\code{}} {}
%    \item{\code{}} {}
%    \item{\code{}} {}     